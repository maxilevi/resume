% !TEX program = xelatex

\documentclass{resume}
%\usepackage{zh_CN-Adobefonts_external} % Simplified Chinese Support using external fonts (./fonts/zh_CN-Adobe/)
%\usepackage{zh_CN-Adobefonts_internal} % Simplified Chinese Support using system fonts

\begin{document}
\pagenumbering{gobble} % suppress displaying page number

\name{Maximiliano Levi}

\basicInfo{
  \email{maxilevi77@gmail.com} \textperiodcentered\ 
  \phone{(+54) 1155299639} \textperiodcentered\ 
  \github[maxilevi]{https://github.com/maxilevi}
  }

\section{\faGraduationCap\ Education}
\datedsubsection{\textbf{Universidad de Buenos Aires (UBA)}, Buenos Aires, Argentina}{2018 -- Present}
\textit{Bachelor's degree} in Computer Science, expected December 2021

\section{\faUsers\ Experience}
\datedsubsection{\textbf{Avature} Buenos Aires, Argentina}{Jan 2018 -- Mar 2019}
\role{Developer}
Worked maintaining a proprietary PHP framework as well as developing new features for the main SaaS application.


\datedsubsection{\textbf{Personal Projects}}{}
\role{OpenGL, C\#, C++}{Project Hedra}
4 year long project, open world 3D game built in C\#, created from scratch using OpenGL and bulletphysics.
\begin{itemize}
  \item Published on the most popular gaming store (\href{https://store.steampowered.com/app/1009960/Project_Hedra/}{Steam Link})
  \item 4 years of consistent development
  \item Built from scratch creating my own engine
\end{itemize}

\role{Python}{airport-system}
Implementation of several graph algorithms (dijkstra, prim, pagerank, bfs, etc) using real airport & flights data
\href {https://github.com/maxilevi/airport-system}{GitHub}
\begin{itemize}
  \item Most common graph algorithms are implemented (eg. Dijkstra, Prim, BFS, DFS, etc)
  \item Some other coding techniques are also shown (eg. Backtracking)
\end{itemize}

\role{NASM, Intel x86}{asm-examples}
A repository with a collection of example programs written in assembly
\href {https://github.com/maxilevi/asm-examples}{GitHub}
\begin{itemize}
  \item Written as a learning excercise
  \item All the routines are documented as to show what is happening
\end{itemize}

\role{JS, webGL}{isosurface}
A sample web demo that showcases different isosurfaces extraction techniques like Marching Cubes & Marchin Tetrahedra
\href {https://github.com/maxilevi/isosurface}{GitHub}

% Reference Test
%\datedsubsection{\textbf{Paper Title\cite{zaharia2012resilient}}}{May. 2015}
%An xxx optimized for xxx\cite{verma2015large}
%\begin{itemize}
%  \item main contribution
%\end{itemize}

\section{\faCogs\ Skills}
\begin{itemize}[parsep=0.5ex]
  \item Programming Languages: C, C++, C#, PHP, Python, Java, SQL
  \item Platform: Linux, Windows
  \item Technologies: OpenGL, webGL, GLSL,
  \item Development: Desktop, Web
\end{itemize}

\section{\faHeartO\ Honors and Awards}
\datedline{\textit{\nth{1} Prize}, Award on xxx }{Jun. 2013}
\datedline{Other awards}{2015}

\section{\faInfo\ Miscellaneous}
\begin{itemize}[parsep=0.5ex]
  \item Blog: http://blog.projecthedra.com
  \item GitHub: https://github.com/username
  \item Languages: English - Fluent (CAE), Spanish - Native speaker
\end{itemize}

\end{document}
